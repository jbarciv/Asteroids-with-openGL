\section{Pruebas}
Las pruebas se dividían en dos bloques: pruebas de juego y de funcionamiento del código. 
Probar el juego fue un proceso entretenido y agradable una vez conseguido que el programa se ejecutara ya que eran fallos muy visuales y resultaba fácil detectar los fallos y su lugar en el código.
Comprobar el funcionamiento del código fue, sin embargo, un poco más tedioso porque el juego no llegaba a ejecutarse y el código no generaba errores, por lo que encontrar los fallos fue un proceso arduo.

Nos hemos podido centrar en el funcionamiento de cada clase que compone el programa aislándolo y comprobando su comportamiento en el juego anulando los demás componentes. Por ejemplo, una vez los asteroides funcionaban correctamente, se establecio su número en cero para poder centrarnos en el \emph{ovni} y el \emph{angel}. Finalmente, cada uno de los integrantes del equipo probó el funcionamiento del juego en su totalidad.