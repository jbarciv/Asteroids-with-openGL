\section{Propuestas de mejora y valoración personal}
Al tener un carácter clásico, el juego resulta sencillo y en ello reside su gracia. Evidentemente, hay infinidad de mejoras o extensiones que se podrían implementar. Sin embargo la principal preocupación del equipo era conseguir la versión básica del juego funcionando para familiarizarse con el entorno de \textsc{OpenGL} y posteriormente, si se tenía el tiempo necesario, se implementaría alguna mejora.

Una vez conseguido el juego en su versión básica como se ha mencionado con anterioridad, a excepción del disparo del \emph{ovni} en dirección a la \emph{nave}. Se decide incorporar un \emph{power up} que recibe el nombre de \emph{Angel} en el juego. Este nuevo elemento se trata de una tetera predefinida en \emph{OpenGL} que aparece después de 40 segundos de juego para otorgar al jugador la posibilidad de conseguir una vida extra si se acerca a ella y la ‘recoge’.

Otras posibles mejoras podrían ser la inclusión de efectos de sonidos, un menú principal y la posibilidad e gestionar usuarios (almenos de forma local) para poder realizar rankings de mejores puntuaciones.