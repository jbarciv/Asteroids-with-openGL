\section{Reparto de Roles}
Para el reparto de tareas se decidió seguir el mismo proceso que para el trabajo anterior. En los primeros pasos de la investigación se realizaron reuniones y recopilación de ideas en las cuales cada integrante del grupo desarrolló poco a poco un boceto de la clase \emph{ObjectList}.

Una vez se consiguió implementar dicha clase, se repartieron las tareas de desarrollo del resto del código entre Linux y Windows para seguir elaborando el programa y realizar pruebas en ambos sistemas operativos. 

Al mismo tiempo se realizaba el diseño de los diagramas UML que servía de apoyo para escribir los códigos de los elementos que conforman el juego.

Cuando se consiguió una versión funcional se repartieron entre el equipo tareas de optimización y solución de errores.

Finalmente, con el programa finalizado, surgió la idea de implementar una nueva clase para añadir un toque original al proyecto. Dos integrantes desarrollaron e implementaron la nueva adición mientras el tercero se ocupaba de la parte documentativa del proyecto. Mientras se manteniendo al día la memoria y se redactaba las descripciones necesarias acerca de este. 